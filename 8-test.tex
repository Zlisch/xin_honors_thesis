\chapter{Test}

We define what it is for a choreograph to be well-formed with the \HOLinline{\HOLFreeVar{G},\HOLFreeVar{Th}\ensuremath{\vdash}\HOLFreeVar{c}\ensuremath{\Box}} relation.

This is a theorem:
\begin{HOLmath}
\ensuremath{\HOLTokenTurnstile}\HOLConst{\ensuremath{\emptyset}},\HOLFreeVar{\ensuremath{\Theta}}\ensuremath{\vdash}\HOLFreeVar{c}\ensuremath{\Box}\;\HOLSymConst{\HOLTokenImp{}}\;\HOLSymConst{\HOLTokenExists{}}\HOLBoundVar{\ensuremath{\tau}}\;\HOLBoundVar{l}\;\ensuremath{\HOLBoundVar{s}\sp{\prime{}}}\;\ensuremath{\HOLBoundVar{c}\sp{\prime{}}}.\;\provtrans{\HOLConst{\ensuremath{\emptyset}}\rhd\HOLFreeVar{c}}{\HOLBoundVar{\ensuremath{\tau}}}{\HOLBoundVar{l}}{\ensuremath{\HOLBoundVar{s}\sp{\prime{}}}\rhd\ensuremath{\HOLBoundVar{c}\sp{\prime{}}}}\;\HOLSymConst{\HOLTokenDisj{}}\;\HOLSymConst{\HOLTokenNeg{}}\HOLConst{not_finish}\;\HOLFreeVar{c}
\end{HOLmath}

The transition relation looks like \HOLinline{\HOLConst{eval_exp}\;\HOLFreeVar{clk}\;\HOLFreeVar{E}\;\HOLFreeVar{exp}}

new test