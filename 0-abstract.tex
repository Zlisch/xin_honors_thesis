\chapter*{Abstract}

Distributed systems are ubiquitous but writing endpoint programs can be 
error-prone since mismatched message sending and receiving can lead to 
errors such as deadlock, where the system indefinitely awaits a message.
Choreography offers a solution by providing a global description of how 
messages are exchanged among endpoints, where message mismatches are disallowed ---
a property called "deadlock-free by design." The global choreography 
is then projected into process models for each endpoint via EndPoint Projection (EPP),
preserving the deadlock-free property.

While many choreography languages focus on message exchange behaviors, few address 
the local computations occurring within endpoints. Most current languages assume local 
computation results or delegate them to external languages. While this offers a reasoning 
ground for studying message exchange behaviours of choreography, when it comes to writing 
a concrete choreography program, the former can only exchange literal values and the 
latter leads to cumbersome code due to the addition of an external computation program which 
typically involves conversions between choreography values and external data types.

Hence in this thesis, we extend Kalas, a state-of-art choreography language with
verified end-to-end compilation, with a local language \richerlang, such that
local computations are handled gracefully within a few lines of codes. Moreover,
it also allows us to formally analyse the message exchange behaviours of choreography 
when local computations are considered. 

We design \richerlang\, as a simply typed lambda calculus with function closure. We use
a functional bis-step semantics with clocks to ensure the evaluation function for
\richerlang\, is total. We prove type soundness for the proposed semantics and typing rules. 
We also provide a strong normalisation proof for \richerlang.

We extend the Kalas'\,\textit{let} transition with \richerlang\, expressions. Besides common data types 
such as integer, string, and boolean, we also add function, pair, and sum types to the 
local computation in choreography . Common operators for our data types are included, such as addition, modulo, 
and negation. An integer-string converter is implemented as well to handle message strings 
in Kalas. Last but not least, we prove the enriched Kalas enjoys progress. We also show
type preservation holds for non-recursive, synchronous transitions.



% An abstract is a very short summary (around 15 lines) of your entire work (that doesn't use citations by convention). There are plenty of examples you can take a look at -- simply take a look at some papers published at top-tier venues, e.g., by your supervisor.

% - choreography diagram; CC, Kalas
% - the meaning of deadlock freedom by design
% - mostly focus on interactions via message passing
% - contribution 1: richerLang: call by value, functional big step semantics with clock to be implemented in HOL4, richer data types, environment semantics; an environmental language model with type theory; and strong normalisation property
% - contribution 2: the enriched choreography, with a simple type theory, Kalas have the safety property of deadlock freedom