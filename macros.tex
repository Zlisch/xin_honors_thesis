
% define your own macros here
\definecolor{vlightgray}{gray}{0.95}
\newcommand{\comptitle}[2]{\multicolumn{1}{c#1}{\cellcolor{vlightgray}#2}}

\newcommand{\Eff} {\ensuremath{\mathit{eff}}}  % example command without arguments
\newcommand{\Pre} {\ensuremath{\mathit{pre}}}  % (again)

% Note that you can easily specify arguments:
% \newcommand{\someMacro}[2] {Argument 1: #1, Argument 2: #2} % example command with two arguments
% you use it via \someMacro{Hello}{World!}


% the following commands are being provided by the amsthm package
% the first parameter states the new environmet's name that can be
% used (due to this definition here) and the second the name that
% will appear in the PDF document
\theoremstyle{plain} %% definition
\newtheorem{definition}{Definition}   % well, a formal definition!
\theoremstyle{plain}
\newtheorem{prop}{Proposition} % like a theorem, but less important or evolved
\newtheorem{lem}{Lemma}        % used within a proof of a theorem
\newtheorem{thm}{Theorem}      % well, a theorem! :) important and evolved
\newtheorem{cor}{Corollary}    % basically either a proposition or theorem,
                               %  but one that follows from another theorem.
\newtheoremstyle{normalstyle}   % Name
  {10pt}        % Space above
  {10pt}        % Space below
  {\normalfont}  % Body font (normal text)
  {}            % Indent amount
  {\bfseries}   % Theorem head font
  {.}           % Punctuation after theorem head
  { }           % Space after theorem head
  {}            % Theorem head spec (can be left empty)
\theoremstyle{normalstyle}
\newtheorem{example}{Example}[section]
% There's a lot you can configure about the appearance. If interested,
% open the manual of amsthm or google for tutorials etc. on that package

% the following add a symbol to the definition environment to make it more
% clear when a definition ends (as there is no difference in fonts!). From:
% https://tex.stackexchange.com/questions/226334/change-a-amsthm-theorem-ending
\newcommand{\xqed}[1]{%
    \leavevmode\unskip\penalty9999 \hbox{}\nobreak\hfill
    \quad\hbox{\ensuremath{#1}}}
\newcommand{\Endofdef}{\xqed{\blacksquare}}
\newenvironment{defn}[1]{%
    \begin{definition}#1}{%
    \Endofdef\end{definition}%
}
\newcommand{\provtrans}[4]{#1 \;\xrightarrow[{#3}]{#2}\;#4}

%\definecolor{sgreen}{rgb}{0,0.3,0}
\newcommand{\api}{applied pi-calculus}
\newcommand{\ccpic}{concurrent constraint pi-calculus}
\newcommand{\ccpi}{CC-Pi calculus}
\newcommand{\efc}{explicit fusion calculus}
\newcommand{\pif}{pi-F}
%\newcommand{\spic}{spi calculus}
%\newcommand{\Api}{Applied pi-Calculus}
%\newcommand{\picalculus}{pi calculus}
%\newcommand{\Picalculus}{Pi calculus}
\newcommand{\psicalculus}{psi-calculus}
\newcommand{\psicalculi}{psi-calculi}
\newcommand{\Psicalculi}{Psi-calculi}
\newcommand{\ve}[1]{\widetilde{#1}}
\newcommand{\0}{\emptyset}
\newcommand{\Set}[2][]{\{ #1 #2 #1 \}}
\newcommand{\POWERSET}{\mathcal{P}}
\newcommand{\powerset}[1]{\POWERSET(#1)}
\newcommand{\POWERFIN}{\POWERSET_{\!\text{fin}}}
\newcommand{\powerfin}[1]{\POWERFIN(#1)}
% \newcommand{\defn}{\;\mathrm{\triangleq}\;}
\newcommand{\apitransarrow}[1]{\goesto{#1}}
\newcommand{\apinottransarrow}[1]{\notgoesto{#1}}
\newcommand{\nottrans}[2]{#1 \; \apinottransarrow{#2}}
\newcommand{\compacttrans}[3]{#1 \apitransarrow{#2} #3}
\newcommand{\trans}[3]{#1 \; \apitransarrow{#2} \; #3}
\newcommand{\otrans}[3]{#1 \; \apitransarrow{#2}_o \; #3}
\newcommand{\attrans}[5]{#1 \; \apitransarrow{#2}_{#5}^{#4} \; #3}
\newcommand{\utrans}[3]{#1 \; \apitransarrow{#2}_U \; #3}
\newcommand{\transpi}[3]{#1 \; \apitransarrow{#2}_\pi \; #3}
\newcommand{\transapi}[3]{#1 \; \apitransarrow{#2}_{A\pi} \; #3}
\newcommand{\transs}[4]{#1~\xrightarrow[#3]{#2}~#4}
\newcommand{\sgto}[2]{\xrightarrow[#2]{#1}}
%\newcommand{\transnn}[4]{#1 \; \goesto{#2\;[#3]} \; #4}
%\newcommand{\transs}[3]{#1 \; \goesto{#2}_s \; #3}
%\newcommand{\transb}[3]{#1 \; \Sgoesto{#2} \; #3}
%\newcommand{\transc}[4]{#1 \; \xrightarrow[#3]{#2} \; #4}
%\newcommand{\transw}[3]{#1 \; \stackrel{#2}{\Longrightarrow} \; #3}
%\newcommand{\transws}[3]{#1 \; \stackrel{#2}{\Longrightarrow}_s \; #3}
%\newcommand{\outunit}[1]{\overline{#1}}

% Send
\newcommand{\sendtok}{\mbox{\rm \textbf{send}}}
\newcommand{\totok}{\mbox{\rm \textbf{to}}}
\newcommand{\outprefix}[2]{\sendtok\,#2\,\totok\,#1}
\newcommand{\outprefixpl}[3]{\outprefix{#1}{#2\sb{#3}}}
% Receive
\newcommand{\recvtok}{\mbox{\rm \textbf{receive}}}
\newcommand{\fromtok}{\mbox{\rm \textbf{from}}}
\newcommand{\intok}{\mbox{\rm \textbf{in}}}
\newcommand{\inprefixpi}[2]{\recvtok\,#2\,\fromtok\,#1}
\newcommand{\inprefixpl}[3]{\recvtok\,#2\,\intok\,\langle #3 \rangle\,\fromtok\,#1}

% Internal choice
\newcommand{\choosetok}{\mbox{\rm \textbf{choose}}}
\newcommand{\fortok}{\mbox{\rm \textbf{for}}}
\newcommand{\intchoiceprefix}[2]{\choosetok\,#2\,\fortok\,#1}
% External choice
\newcommand{\choosestok}{\mbox{\rm \textbf{chooses}}}
\newcommand{\ortok}{\mbox{\rm \textbf{or}}}

\newcommand{\extchoice}[3]{#1\,\choosestok\,\HOLConst{T}:\,#2\,\ortok\,\HOLConst{F}:#3}


\newcommand{\outprefixempty}[1]{\overline{#1}}
\newcommand{\outprefixpiF}[1]{\overline{#1}}
\newcommand{\outprefixcc}[2]{\outprefix{#1}{#2}}
\newcommand{\brinlabelempty}[1]{\underline{\text{\textrm{?}}#1}}
\newcommand{\brinlabel}[2]{\brinlabelempty{#1}\: #2}
\newcommand{\brbinlabel}[2]{\brinlabelempty{#1}(#2)}
\newcommand{\broutlabelempty}[2][p]{\overline{{\textrm{!}#2}}}
\newcommand{\broutlabel}[3][p]{\broutlabelempty[#1]{#2}\: #3}
\newcommand{\brbout}[4][p]{\broutlabelempty[#1]{#3}\:(\mathbf{\nu}#2) #4}
%\newcommand{\brbout}[3]{\broutlabelempty{#2}\:(\mathbf{\nu}#1) #3}
\newcommand{\binlabel}[2]{\underline{#1}(#2)}
\newcommand{\inlabel}[2]{\underline{#1}\: #2}
\newcommand{\inlabelpi}[2]{\leftarrow#1: #2}
\newcommand{\inlabelempty}[1]{\underline{#1}}
\newcommand{\inlabelpiF}[1]{\underline{#1}}
\newcommand{\inlabelcc}[2]{\inprefixcc{#1}{#2}}
\newcommand{\inlabelccb}[2]{\inprefixcc{#1}{(\nu #2)#2}}
\newcommand{\inlabellate}[3]{\underline{#1}(#3)}
\newcommand{\inlabelpm}[3]{\underline{#1}(\lambda #2) #3}
\newcommand{\outlabel}[2]{\rightarrow{}#1:#2}
\newcommand{\coutlabel}[3]{#1 \rightarrow #2: #3}
\newcommand{\intchlabel}[3]{\coutlabel{#1}{#2}{#3}}
\newcommand{\cinlabel}[3]{#2 \leftarrow #1: #3}
\newcommand{\outlabelpiF}[1]{\overline{#1}}
\newcommand{\outlabelempty}[1]{\overline{#1}}
\newcommand{\outlabelcc}[2]{\outprefixcc{#1}{#2}}
\newcommand{\outlabelccb}[2]{\outprefixcc{#1}{(\nu #2)#2}}
%\newcommand{\innn}[3]{\underline{#1} (#3)}
\newcommand{\inprefixshort}[2]{\underline{#1}(#2)}
\newcommand{\inprefixempty}[1]{\underline{#1}}
\newcommand{\inprefixpiF}[1]{\underline{#1}}
\newcommand{\inprefixcc}[2]{\underline{#1} #2}
\newcommand{\inprefixsym}[2]{#1(#2)}
\newcommand{\cendpoint}[4]{(#1, #2, #3) \rhd #4}
\newcommand{\pendpoint}[5]{(#1, #2, #3, #4) \rhd #5}
\newcommand{\brinprefix}[3]{\underline{\text{\textrm{?}}#1}(\lambda #2) #3}
\newcommand{\grossarrow}{\mathrel{\mathtt{-\!\!\!\!>}}}
\newcommand{\commprefix}[4]{#1 . #2  \grossarrow #3 . #4}
\newcommand{\commprefixsimp}[2]{#1 \grossarrow #2}
\newcommand{\selprefix}[3]{#1 \grossarrow #3[#2]}
\newcommand{\tell}[1]{\mathbf{tell}\; #1}
\newcommand{\ask}[1]{\mathbf{ask}\; #1}
%\newcommand{\lin}[3]{\underline{#1}(\lambda #2) #3}
\newcommand{\bout}[3]{\overline{#2}\:(\mathbf{\nu}#1) #3}
%\newcommand{\sbout}[3]{\overline{#1}(\mathbf{\nu}#2)#3}
\newcommand{\fuselabel}[2]{?#1\mbox{\scriptsize{=}}#2}
\newcommand{\fusion}[2]{#1\mbox{\scriptsize{=}}#2}
\newcommand{\datum}[1]{\langle #1\rangle}
%\newcommand{\oout}[2]{\mathbf{\exists}#2\overline{#1}\langle #2 \rangle}
%\newcommand{\inp}[2]{#1\langle #2 \rangle}
%\newcommand{\sinp}[3]{#1\:\lambda #2 #3}
%\newcommand{\sout}[2]{\overline{#1}#2}
%\newcommand{\binp}[2]{#1(#2)}
%\newcommand{\sbn}[1]{{\rm sbn}(#1)}
%\newcommand{\conds}{\triangleleft}
%\newcommand{\vareq}{\equiv_{\rm V}}
%\newcommand{\equivc}{\equiv_{c}}
%\newcommand{\equivn}{\equiv_{n}}
%\newcommand{\equivr}{\equiv_{r}}
%\newcommand{\equivs}{\equiv_{s}}
%\newcommand{\lequal}[1]{=_{#1}}
%\newcommand{\equations}{\mbox{\rm \bf \sf \small E}}
%\newcommand{\aframe}[1]{\mathcal{F}(#1)}
%\newcommand{\sframe}[1]{\mathcal{F}(#1)}
%\newcommand{\tframe}[1]{\mathcal{F}(#1)}
%\newcommand{\cframe}[1]{\mathcal{F}_\Phi(#1)}
%\newcommand{\fc}[1]{\mathcal{C}_F[#1]}
\newcommand{\frameof}[1]{\mathcal{F}(#1)}
%\newcommand{\ag}[1]{\mathcal{A}(#1)}
%\newcommand{\fradd}{\cup}
\newcommand{\frames}{\,\rhd\,}
%\newcommand{\sol}[1]{sol(#1)}
%\newcommand{\al}[2]{(\nu #2=#1)}
%\newcommand{\alias}[3]{\textbf{alias } \{{}^{#1}\!/\!{}_{#2}\} \textbf{ in }#3}
%\newcommand{\letin}[3]{\textbf{let } #2 = #1 \textbf{ in } #3}
%\newcommand{\sat}[2]{Sat_{#1}(#2)}
\newcommand{\res}[1]{(\mathbf{\nu}#1)}
%\newcommand{\resvar}[2]{(\mathbf{\nu}#1:#2)}
%\newcommand{\scope}[1]{\mathbf{\nu}#1}
%\newcommand{\fresh}[1]{\res{#1}}
\newcommand{\freshin}{\#}
%\newcommand{\notfreshin}{}
%\newcommand{\ground}[1]{ground(#1)}
%\newcommand{\constrains}{\triangleright}
%\newcommand{\lconstrains}{\triangleleft}
%\newcommand{\mappings}{\triangleright}
%\newcommand{\hooks}{\triangleleft}
%\newcommand{\och}{\wedge}
%\newcommand{\logicify}[1]{logicify(#1)}
\newcommand{\iftok}{\mbox{\textrm{\textbf{if}}}}
\newcommand{\thentok}{\mbox{\textrm{\textbf{then}}}}
\newcommand{\elsetok}{\mbox{\textrm{\textbf{else}}}}
\newcommand{\ifthel}[3]{\mbox{\iftok{} \ensuremath{#1} \thentok{} \ensuremath{#2} \elsetok{} \ensuremath{#3}}}
\newcommand{\fixc}[2]{\ensuremath{\mu #1. #2}}
\newcommand{\fixcall}[1]{#1}
\newcommand{\ifthen}[2]{\mbox{\iftok{} \ensuremath{#1} \thentok{} \ensuremath{#2}}}
% \newcommand{\letin}[1]{\mbox{\rm\ensuremath{\textbf{let } #1 \textbf{ in }}}}
\newcommand{\letin}[1]{\mbox{\textrm{\textbf{let }}\ensuremath{#1}\textrm{\textbf{ in }}}}
\newcommand{\letrecin}[2]{\mbox{\rm\ensuremath{\textbf{letrec } #1 \textbf{ in }
#2}}}
\newcommand{\closure}[2]{\mbox{\rm\ensuremath{\textbf{Cl } #1 \textbf{ in }
#2}}}
\newcommand{\letreconly}{\mbox{\rm\ensuremath{\textbf{letrec }}}}
\newcommand{\letonly}{\mbox{\rm\ensuremath{\textbf{let }}}}
\newcommand{\ci}[2]{#1:#2} % Case item
\newcommand{\cit}[1]{#1:} % Case item
%\newcommand{\caseelse}[2]{\textbf{case } #1 \textbf{ else } #2}
%\newcommand{\caseoelse}[2]{\textbf{case } #1 \text{ }[\textbf{else } #2]}
\ifdefined\case
\renewcommand{\case}[1]{\mbox{\rm\ensuremath{\textbf{case } #1}}}
\else
\newcommand{\case}[1]{\mbox{\rm\ensuremath{\textbf{case } #1}}}
\fi
\newcommand{\casesep}{\mathrel{[\hspace{-0.1ex}]}} %black box please?
\newcommand{\casen}{\mbox{\rm\ensuremath{\textbf{case\ }}}}
\newcommand{\boldword}[1]{\mbox{\rm\ensuremath{\textbf{#1}}}}
\newcommand{\caseonly}{\boldword{case}}
%\newcommand{\casesprod}[1]{\textbf{case}\casesep #1}
\newcommand{\pll}{\;|\;}
%\newcommand{\cpll}{\:\parallel\:}
%\newcommand{\vect}[1]{\langle #1 \rangle}
%\newcommand{\typeass}[3]{#1:\vect{#2;#3}}
%\newcommand{\tass}[2]{#1:#2}
%\newcommand{\typeassv}[3]{\tilde{#1}:\vect{\tilde{#2};\tilde{#3}}}
%\newcommand{\tassv}[2]{\tilde{#1}:\tilde{#2}}
%\newcommand{\oftype}{\in}
\newcommand{\match}[3]{[#1\!\!=\!\!#2]#3}
\newcommand{\mismatch}[3]{[#1\!\!\neq\!\!#2]#3}
\newcommand{\nil}{\ensuremath{\mathbf{0}}}
%\newcommand{\config}[2]{#1 \,\rhd\, #2}
%\newcommand{\configs}[3]{[#1,#2]\# #3}
%\newcommand{\extend}{+}
%\newcommand{\subtype}{<:}
%\newcommand{\tuple}[2]{\langle #1 , #2 \rangle}
%\newcommand{\exposes}{\downarrow}
%\newcommand{\exposesw}{\Downarrow}
%\newcommand{\wbarbedequivalent}{\cong}
%\newcommand{\wbarbedcongruent}{\cong^c}
%\newcommand{\C}{\mathcal{C}}
%\newcommand{\B}{\mathcal{B}}
%\newcommand{\D}{\mathcal{D}}
%\newcommand{\E}{\mathbf{E}}
%\newcommand{\G}{\mathcal{G}}
%\newcommand{\I}{\mathcal{I}}
%\newcommand{\N}{\mathcal{N}}
%\newcommand{\V}{\mathcal{V}}
%\newcommand{\T}{\mathcal{T}}
\newcommand{\R}{\mathrel{\mathcal{R}}}
%\newcommand{\X}{\mathcal{X}}
%\newcommand{\sembrack}[1]{[\![#1]\!]}
%\newcommand{\channel}{\mathcal{C}}
%\newcommand{\instanceof}[1]{\rel{\surd_{\!\!{#1}}}}
%\newcommand{\instmap}[2]{\ldbr #1 \rdbr_{#2}}
%\newcommand{\instmapd}[3]{\ldbr #1 \rdbr_{#3}^{#2}}
%\newcommand{\equiinstanceof}{\stackrel{=}{\surd}}
\newcommand{\infers}{\vdash_s}
\newcommand{\simplies}{\leq}
%\newcommand{\ssimplies}{\simplies_s}
%\newcommand{\xxximplies}{\simeq\hspace*{-3pt}>}
\newcommand{\sequivalent}{\simeq}
\newcommand{\ACequivalent}{\mathrel{\simeq_{\mathsf{AC}}}}
%\newcommand{\ssequivalent}{\simeq_s}
%\newcommand{\nssequivalent}{\nLeftrightarrow_s}
%\newcommand{\consistent}[1]{\odot(#1)}
%\newcommand{\sidecondition}[1]{\small{$1}}
%\newcommand{\subst}[2]{\{{}^{#1}\!/\!{}_{#2}\}}
%\newcommand{\ssubst}[2]{\{{}^{#1}\!/\!{}_{#2}\}}
%\newcommand{\isubst}[2]{\{{}^{#1}\!/\!{}_{#2}\}}
\newcommand{\subst}[2]{[#1 :=\! #2]}
\newcommand{\asubst}[2]{\{{}^{#2} / {}_{#1}\}}
%\newcommand{\osubst}[2]{{\lsubst{#1}{#2}}_{\mbox{\rm o}}}
%\newcommand{\fsubst}[2]{\circ \lsubst{#1}{#2}}
%\newcommand{\ev}[1]{\mathit{ev}(#1)}
%\newcommand{\n}{\mbox{{\rm n}}}
%\newcommand{\carries}[2]{\mbox{$#1$ carries $#2$}}
%\newcommand{\evs}[2]{\mathit{ev}_{#2}(#1)}
%\newcommand{\constraints}[1]{\mathit{cons}(#1)}
%\newcommand{\maps}[1]{\mathit{maps}(#1)}
%\newcommand{\cons}[1]{\mathit{cons}(#1)}
%\newcommand{\satisfies}{\models}
%\newcommand{\ldb}{\mbox{[\![}}
%\newcommand{\vars}{\mathit{v}}
%\newcommand{\atoms}{\mathit{atoms}}
%\newcommand{\fm}[1]{fm(#1)}
%\newcommand{\map}[2]{\isubst{#1}{#2}}
%\newcommand{\fsubj}[1]{fsubj(#1)}
\newcommand{\sym}[1]{\ensuremath{\mathsf{#1}}}
%\newcommand{\translAtoE}[1]{\ldbr #1 \rdbr}
%\newcommand{\translPolytoE}[1]{\ldbr #1 \rdbr}
%\newcommand{\enc}[1]{\ldbr #1\rdbr}
%\newcommand{\inred}[1]{\textcolor{red}{#1}}
%\newcommand{\fdom}[1]{\mbox{\rm dom($#1$)}}
%\newcommand{\state}{\textbf{state }}
%\newcommand{\inv}{\textbf{inv }}
%\newcommand{\statements}{\mathcal{S}}
%\newcommand{\bad}{\mathbf{bad}}
%\newcommand{\wf}[1]{\mathbf{wf}(#1)}
%\newcommand{\true}{\rm true}
%\newcommand{\false}{\rm false}
\newcommand{\ftimes}{\otimes}
%\newcommand{\fplus}{\oplus}
%\newcommand{\trip}[1]{\ldbr #1 \rdbr_{\text{strip}}}
%\newcommand{\nf}[1]{\mbox{\rm nf}(#1)}
%\newcommand{\cond}[1]{\mbox{\rm co}(#1)}
%\newcommand{\states}[1]{\mbox{\rm st}(#1)}
%\newcommand{\naked}[1]{\mbox{\rm unfr}(#1)}
\newcommand{\unit}{{\bf 1}}
\newcommand{\identity}{{\bf I}}
%\newcommand{\Ch}{\mbox{\rm\textbf{C}}}
%\newcommand{\Chtype}[1]{#1:\Ch}
%\newcommand{\ceq}{\stackrel{\texttt{{\rm\tiny C}}}{\mapsto}}
%\newcommand{\sameCh}{\mbox{\rm\texttt{sameCh}}}
\newcommand{\framepair}[2]{(\nu #1)#2}
%\newcommand{\statcomb}{\cup}
%\newcommand{\sch}{\stackrel{{\hspace{.05ex}\hbox{\ensuremath{\mathbf{.}}}}}{\rightarrowtail}}
\newcommand{\sch}{\stackrel{{\hspace{.05ex}\hbox{\ensuremath{\mathbf{.}}}}}{\leftrightarrow}}
\newcommand{\chcon}{\stackrel{{\hspace{.05ex}\hbox{\ensuremath{\mathbf{.}}}}}{\rightarrow}}
%\newcommand{\stmn}[1]{\{#1\}}
%\newcommand{\framedtrans}[5]{#1: \framedtransempty{#2}{#3}{#4}{#5}}
\newcommand{\framedtrans}[4]{ #1 \frames \trans{#2}{#3}{#4}}
\newcommand{\framedotrans}[4]{ #1 \frames \otrans{#2}{#3}{#4}}
\newcommand{\framedutrans}[4]{ #1 \frames \utrans{#2}{#3}{#4}}
\newcommand{\wframedtrans}[4]{#1 \frames \wtrans{#2}{#3}{#4}}
\newcommand{\frnames}[1]{\ve{b}_{#1}}
\newcommand{\frass}[1]{\Psi_{\!{#1}}}
%\newcommand{\topterms}[1]{\sym{tterms}{#1}}
%\newcommand{\nameset}{\mbox{$\cal N$}}
%\newcommand{\varset}{\mathcal{V}}
%\newcommand{\chanset}{\mathsf{Chan}}
\newcommand{\wtrans}[3]{#1 \; \Goesto{#2} \; #3}
\newcommand{\wtransapi}[3]{#1 \; \Goesto{#2}_{A\pi} \; #3}
\newcommand{\startrans}[3]{#1~\goesto{#2}~\hspace*{-0.5em}*~#3}
\newcommand{\transf}[3]{#1\; \goesto{#2}_F \; #3}
\newcommand{\transcc}[3]{#1\; \goesto{#2}_{cc} \; #3}
\newcommand{\hole}{{[\cdot]}}
\newcommand{\outbarb}[1]{\overline{#1}}
\newcommand{\inbarb}[1]{\underline{#1}}
\newcommand{\wbarbcong}{\approx_b}
\newcommand{\wbarbbisim}{\underset{{\rm barb}}{\wbisim}}
\newcommand{\barbbisim}{\underset{{\rm barb}}{\bisim}}
\newcommand{\exposes}[1]{\downarrow_{#1}}
\newcommand{\nexposes}[1]{\not\downarrow_{#1}}
\newcommand{\wexposes}[1]{\Downarrow_{#1}}
\newcommand{\redtrans}[2]{\trans{#1}{}{#2}}
\newcommand{\apiredtrans}[2]{\transapi{#1}{}{#2}}
\newcommand{\wredtrans}[2]{\wtrans{#1}{}{#2}}
\newcommand{\apiwredtrans}[2]{\wtransapi{#1}{}{#2}}

\newcommand{\sbout}[3]{\overline{#1}(\mathbf{\nu}#2)#3}
\newcommand{\lsubst}[2]{[#2 := #1]}
\newcommand{\wtranss}[4]{#1~\xRightarrow[#3]{#2}~#4}
\newcommand{\chan}{\operatorname{ch}}
\newcommand{\bchan}[1][]{\operatorname{bc}^{#1}}
\newcommand{\bchani}{\bchan[i]}
\newcommand{\bchano}{\bchan[o]}


\newcommand{\pass}[1]{\llparenthesis #1\rrparenthesis}
\newcommand{\allphi}[1]{[#1]}
\newcommand{\swapping}[2]{(#1\;#2)}
\newcommand{\supp}[1]{\mathsf{n}(#1)}
%\newcommand{\defn}{\stackrel{\text{def}}{=}}
\newcommand{\barbcond}[1]{\varphi_{#1}}
\newcommand{\constr}[1]{\{\!\!| #1 |\!\!\}}
\newcommand{\true}{\mathrm{\bf true}}
\newcommand{\false}{\mathrm{\bf false}}
\newcommand{\sol}[1]{{\rm sol}(#1)}
\newcommand{\cimplies}{\Rightarrow}
\newcommand{\lequal}[1]{=_{#1}}
\newcommand{\N}{\mathcal{N}}
\newcommand{\terms}{{\rm\bf T}}
\newcommand{\pats}{{\rm\bf X}}
\newcommand{\conditions}{{\rm\bf C}}
\newcommand{\contexts}{\mathcal{C}}
\newcommand{\chole}{\lbrack\;\rbrack}
\newcommand{\holes}[1]{H(#1)}
\newcommand{\PPR}{\mathrm{ppr}}
\newcommand{\ppr}[1]{\PPR(#1)}
\newcommand{\CONDS}{\mathrm{conds}}
\newcommand{\conds}[1]{\CONDS(#1)}
\newcommand{\cfill}[2]{#1\lbrack#2\rbrack}
\newcommand{\assertions}{{\rm\bf A}}
\newcommand{\processes}{{\rm\bf P}}
\newcommand{\provenances}{\Pi}
\newcommand{\labels}{{\rm\bf L}}
\newcommand{\nomtyp}{{\rm\bf X}}
\newcommand{\nameset}{\mathcal{N}}
\newcommand{\freq}{{\rm\bf Freq}}
\newcommand{\ip}{{\rm\bf Ip}}
\newcommand{\locchan}[2]{#1@#2}
\newcommand{\equat}[2]{\pass{#1=#2}}
\newcommand{\pair}[2]{\sym{t_2}(#1, #2)}
\newcommand{\fst}[1]{\sym{\pi_1}(#1)}
\newcommand{\snd}[1]{\sym{\pi_2}(#1)}
\newcommand{\vars}{\mathit{v}}
\newcommand{\chanset}{\mathsf{Chan}}
\newcommand{\piinstance}{{\rm\bf Pi}}
\newcommand{\encodepi}[1]{\ldbr #1\rdbr_\piinstance{}}
\newcommand{\trisim}[3]{#1\; \frames\; #2 \bisim #3}
\newcommand{\trisims}[3]{#1 \frames #2 \bisim #3}
\newcommand{\trisimsub}[3]{#2 \bisim_{#1} #3}
\newcommand{\wtrisim}[3]{#1\; \frames\; #2 \wbisim #3}
\newcommand{\wtricong}[3]{#1\; \frames\; #2 \wcong #3}
\newcommand{\bisimpi}[2]{#1 \bisim_\pi #2}
\newcommand{\dotsim}[1][\Psi]{\mathrel{\dot{\sim}_{#1}}}
\newcommand{\cbisim}{\mathrel{\underset{{\rm ctx}}{\bisim}}}
\newcommand{\wcbisim}{\mathrel{\underset{{\rm ctx}}{\wbisim}}}
\newcommand{\lbisim}{\bisim^*}
\newcommand{\tricong}[3]{#1 \frames #2 \sim #3}
\newcommand{\lcong}{\sim^*}
\newcommand{\abs}[1]{\lvert{#1}\rvert}
\newcommand{\weakcong}{\approx}
\newcommand{\swbisim}{\underset{{\rm smp}}{\wbisim}}
\newcommand{\sbisim}{\underset{{\rm sym}}{\bisim}}
\newcommand{\taubisim}[3]{#1 \frames #2 \underset{{\rm tau}}{\wbisim} #3}
\newcommand{\freev}[1]{\mathrm{fv}(#1)}
\newcommand{\freep}[1]{\mathrm{fp}(#1)}
\newcommand{\writtenv}[1]{\mathrm{wv}(#1)}
\newcommand{\boundv}[1]{\mathrm{bv}(#1)}
\newcommand{\store}[1]{\mathrm{store}(#1)}
\newcommand{\connect}{\text{FIXME}}
\newcommand{\conin}{\stackrel{{\hspace{.05ex}\hbox{\ensuremath{\mathbf{.}}}}}{\succ}}
\newcommand{\conout}{\stackrel{{\hspace{.05ex}\hbox{\ensuremath{\mathbf{.}}}}}{\prec}}
\newcommand{\BC}{\mathrm{bc}}
\newcommand{\UC}{\mathrm{uc}}
\newcommand{\bc}[1]{\BC(#1)}
\newcommand{\uc}[1]{\UC(#1)}
\newcommand{\kwd}[1]{\mbox{\rm\ensuremath{\textbf{#1}}}}
\newcommand{\nme}[1]{\mbox{\rm\ensuremath{\textit{#1}}}}
\newcommand{\matchop}[2]{\operatorname{\text{$[#1 = #2]$}}}
\newcommand{\stores}{\mathcal{S}}
\newcommand{\prettyindent}{\quad}
\newcommand{\replicationpretty}[1]{\ensuremath{\rep #1}}
\newcommand{\matchpretty}[3]{%
  \begin{array}[t]{@{}l@{}}%
     {}\matchop{#1}{#2}%
  \\ {}\prettyindent{}#3%
  \end{array}%
}
\newenvironment{pretty}{%
  \begin{array}[t]{@{}l@{}}%
}{%
  \end{array}%
}
\newenvironment{groupingpretty}{%
  \begin{pretty}%
     \big(\begin{pretty}%
}{%
      \end{pretty}%
  \\ \big)%
  \end{pretty}%
}
% use: def \qquad\begin{linenumbers}(1)\\(2)\\\ldots\end{linenumbers}
%\newenvironment{linenumbers}{%
%  \begin{tabular}[t]{@{}r@{}}%
%}{%
%  \end{tabular}%
%}

\newcommand{\setsep}{\mathrel{:}}
\newcommand{\numbers}{\mathbb{N}}
\newcommand{\connected}[2]{#1 \rightsquigarrow #2}

\newcommand{\msetof}[1]{\mathtt{multiset\_of}\ #1}

\newcommand{\prindent}{\hspace*{.4cm}}
\newcommand{\parok}[1][]{\textsc{ADMIT}_{\textnormal{#1}}}
\newcommand{\priook}[1][]{\textsc{HIGHEST}_{\textnormal{#1}}}
\newcommand{\reliable}{{\tt reliable}}
\newcommand{\hasprio}{\operatorname{hasprio}}
%\newcommand{\PRIO}{{:}}
%\newcommand{\Prio}[2]{#1 \PRIO #2}
 \newcommand{\PRIO}{\operatorname{{\tt has\_prio}}}
 \newcommand{\Prioof}{\operatorname{{\tt PRIO }}}
 \newcommand{\Prio}[2]{{\tt prio}(#1)= #2}

\newcommand{\semb}[1]{\llbracket #1 \rrbracket}
\newcommand{\instancename}[1]{\textbf{#1}}
\newcommand{\INSTANCE}[2]{
  \[\boxed{
     \begin{array}{c}
        \instancename{#1} \\\hline
          #2
      \end{array}
     }
   \]
  }

\newcommand{\ARRAYOF}[1]{\begin{array}[t]{l} #1 \end{array}}
\newcommand{\instance}[2]{\INSTANCE{#1}{\ARRAYOF{#2}}}
\newcommand{\instanceFrom}[3]{\INSTANCE{#1}{
    \begin{array}{l}
      \text{Everything as in \instancename{#2} except:}\\
      #3
    \end{array}}}

\newcommand{\instanceTwo}[4][\quad]{\INSTANCE{#2}{\ARRAYOF{#3}#1\ARRAYOF{#4}}}

\newcommand{\instanceTwoFrom}[5][\quad]{\instanceFrom{#2}{#3}{\ARRAYOF{#4}#1\ARRAYOF{#5}}}

\newcommand{\True}{\textit{True}}
\newcommand{\Tr}{\textit{Tr}}
\newcommand{\transSys}[1][]{{\longrightarrow}_{\textnormal{#1}}}

\newcommand{\GAP}{0.3em}

% Snoopy Cache macros
\newcommand{\wmem}[3]{\mbox{\rm\ensuremath{w(#1, #2, #3)}}}
\newcommand{\rmem}[2]{\mbox{\rm\ensuremath{r(#1, #2)}}}
%\newcommand{\hit}[1]{\mbox{\rm\ensuremath{\textsf{hit}(\textit{#1})}}}
\newcommand{\hit}[1]{\mbox{\rm\ensuremath{\textsf{hit}(#1)}}}
%\newcommand{\miss}[1]{\mbox{\rm\ensuremath{\textsf{miss}(\textit{#1})}}}
\newcommand{\miss}[1]{\mbox{\rm\ensuremath{\textsf{miss}(#1)}}}
\newcommand{\copies}[1]{\mbox{\rm\ensuremath{\textsf{copies}(#1)}}}
\newcommand{\commit}[2]{\mbox{\rm\ensuremath{\textsf{commit}(#1, #2)}}}
\newcommand{\wass}[3]{\mbox{\rm\ensuremath{\textsf{write}(#1, #2, #3)}}}
\newcommand{\setmodified}[1]{\mbox{\rm\ensuremath{\textsf{SetModified}(#1)}}}
\newcommand{\setexclusive}[1]{\mbox{\rm\ensuremath{\textsf{SetExclusive}(#1)}}}
\newcommand{\setshared}[1]{\mbox{\rm\ensuremath{\textsf{SetShared}(#1)}}}
\newcommand{\setinvalid}[1]{\mbox{\rm\ensuremath{\textsf{SetInvalid}(#1)}}}
\newcommand{\modified}[1]{(#1, \kwd{modified})}
\newcommand{\exclusive}[1]{(#1, \kwd{exclusive})}
\newcommand{\shared}[1]{(#1, \kwd{shared})}
\newcommand{\invalid}[1]{(#1, \kwd{invalid})}
\newcommand{\iam}[2]{\mbox{\rm\ensuremath{#1\ \textsf{is}\ #2}}}
\newcommand{\INVOKE}{\text{\rm\textbf{run}}}
\newcommand{\invoke}[1]{\INVOKE\;#1}
\newcommand{\termcombine}[2]{#1\langle#2\rangle}
\newcommand{\invokep}[2]{\text{\rm\textbf{run}}\; \termcombine{#1}{#2}}
\newcommand{\clausep}[4]{#1 (\lambda #2)#3 \Leftarrow #4}
\newcommand{\clause}[2]{#1 \Leftarrow #2}
\newcommand{\assc}[1]{\{#1\}}
\newcommand{\clauses}{\mbox{\bf Cl}}
\newcommand{\charAss}[2]{\Psi^{\clause{#1}{#2}}}
\newcommand{\hobisim}{\bisim^{\mbox{\sc \tiny ho}}}
\newcommand{\varss}{\textsc{vars}}
\newcommand{\Match}{\textsc{match}}
\newcommand{\snameset}[1][]{\mathcal{N}_{{#1}}}

\newcommand{\SORT}{\textsc{sort}}
\newcommand{\sort}[1]{\SORT(#1)}
\newcommand{\sortset}{\mathcal{S}}
\newcommand{\namesorts}{\mathcal{S_{N}}}
\newcommand{\CanSend}{\mathrel{\overline{\propto}}}
\newcommand{\CanReceive}{\mathrel{\underline{\propto}}}
\newcommand{\CanSubstitute}{\mathrel{\Yleft}}
\newcommand{\Restrictionsorts}{\sortset_\nu}
\newcommand{\pad}{\mathsf{pad}}
\newcommand{\unpad}{\mathsf{unpad}}
\newcommand{\final}{\mathsf{final}}
\newcommand{\intermediate}{\mathsf{intermediate}}
\newcommand{\projectonly}{\ensuremath{\mathsf{pr}}}
\newcommand{\projecte}[2]{\projectonly_{#1}(\epsilon,{#2})}
\newcommand{\projects}[3]{\projectonly_{#1}({#2},{#3})}
\newcommand{\project}[2]{\projectonly_{#1}(\gamma,{#2})}
\newcommand{\procsof}[1]{\mathsf{procs}({#1})}
\newcommand{\onlysplitsel}{\ensuremath{\mathsf{sp}}}
\newcommand{\splitsel}[2]{\onlysplitsel_{#1}({#2})}
\newcommand{\eplang}{\textsc{Endpoint}}
\newcommand{\payloadlang}{\textsc{Payload}}
\newcommand{\escope}{\epsilon}
\newcommand{\scope}{\gamma}
